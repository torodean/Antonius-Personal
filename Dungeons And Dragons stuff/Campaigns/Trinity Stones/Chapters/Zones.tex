\chapter{The Zones}

\section{The Statu Peninsula}

The Statu peninsula (commonly referred to as just Statu) is largly un-mapped area. As the name suggests, Statu is a peninsula that cannot be accessed from foot to the north. It contains very dense forestation and mountainous regions to the northern area that are not possible to travel through. The south is colonized by Aurushire where a small number of people settle. There are generally a small number of visitors to Statu, save those who are trading and working with the mines to the East of Aurushire. The best way to access Statu is through the Aurushire port.

To the West of Aurushire, there is a large forest that is significantly less dense than the northern regions. To the north of this is very dense un-explored forest and mountainous regions. Directly north of Aurushire is The Pluvian Forest. This region is not very often visited. Strange occurances pertaining to what some call 'strange spatial events' occur in The Pluvian Forest which generally ward off visitors. To the southeast of Aurushire, there is a hostile tribe of river Naga which attack travelers and visitors on sight. Because of this, the area to the southeast and the forest regions to the East of Aurushire are unexplored. What is known of this region is that there is a large valley that leads from the Naga camps into the mountain/forest region.

\section{Aurushire}

Aurushire (also commonly referred to as Goldshire) is a small village located on the southern end of the Statu peninsula. Just East of Aurushire is a large mountain containing mine shafts used for gold mining. It is not well known, other by those who work in the mines, but strange things happen within these mines where mined gold will spontaneously be regrown after being removed from the mines. This phenomenon is where Aurushire (Goldshire) retreived its name from and is thought to be related to the spacial occurrences reported in The Pluvian Forest. 

\subsection{NPC's}

Aurushire is full of NPC characters. Many of them are just normal workers or village folk, however some serve an important purpose.

\begin{monsterbox}{Arryn}
	\begin{hangingpar}
		\textit{Halfling Wizard, Neutral Good}
	\end{hangingpar}
	\dndline%
	\basics[%
	armorclass = 17,
	hitpoints  = 172,
	speed      = 40 ft
	]
	\dndline%
	\stats[
	STR = \stat{12}, % This stat command will autocomplete the modifier for you
	DEX = \stat{12},
	CON = \stat{16},
	INT = \stat{20},
	WIS = \stat{18},
	CHA = \stat{18}
	]
	\dndline%
	\details[%
	% If you want to use commas in these sections, enclose the
	% description in braces.
	% I'm so sorry.
	languages = {Common, Elvish, Dwarvish, Gnomish, Halfling, Orc, Pandaren},
	]
	\dndline%
	\begin{monsteraction}[Mystical Senses]
		If a target tries to deceive you, it must make a DC19 deception saving throw.
	\end{monsteraction}	
	\begin{monsteraction}[Water sight]
		You can see clearly up to two miles over water.
	\end{monsteraction}
	\monstersection{Actions}
	\begin{monsteraction}[Bladesinger: Extra Attack]
		You can attack twice on your turn.
	\end{monsteraction}
	\begin{monsteraction}[Hold Monster]
		A creature you can see within 90 feet must succeed on a wisdom saving throw or be paralyzed for 1 minute. Target can make a wisdom saving throw at the end of each of it's turns to end the spell.
	\end{monsteraction}
	\begin{monsteraction}[Magic Jar Master]
		Allows Arryn to use Magic Jar on a nearby jar instantly and without being in a jar himself.
	\end{monsteraction}
	\begin{monsteraction}[Magic Jar]
		Long description, look it up.
	\end{monsteraction}
	\monstersection{Description/Information}
	Arryn is the first NPC most encounter when entering Aurushire from sea. He is the watcher of the ports and lives on a small farm right at docks where all sea traffic enters from.
\end{monsterbox}

\section{Rem Silva}

Rem Silvia is the name of the Forest that is located to the West of Aurushire. This forest, along with The Pluvian Forest often experiences strange spacial phenomenon. Unlike The Pluvian Forest, temporal phenomenon have also commonly been reported as occurring in this region. The occurrences of these random phenomenon do not appear marginally as often or as strong as in The Pluvian Forest which makes this region of particular interest to experienced hunters.

The same strange effects that can occur in the Pluvian forest can also occur here in Rem Silva only less frequently. As a DM, you can periodically roll a 1d20 to see if any of the irregular effects from The Pluvian Forest will also occur here. Subtract 5 from each DC throw to see if the effects occur in this region.

This area is full of a large number of creatures, from large spiders, to night elfs. Due to the trinity stone effects, many creatures not belonging to this region also appear here. 

\begin{commentbox}{Random Encounters}
	At any time, a party traveling through this area could run into a creature. To determine a random encounter you can roll 1d100 and choose what the party will encounter. If the roll is odd, choose a miscellaneous creature from Appendix A of the monster manual (page 317-337 in monster manual, roll 1d20 to decide page of creature to choose), otherwise if the roll is even, choose from the table below. Similarly, Page 97 of Xanathars may be useful for creating random encounters.
	\begin{description}
		\item[1-10:] Displacer Beast
		\item[11-20:] Basilisk
		\item[21-25:] Dinosaur (page 79-80 of monster manual)
		\item[26-30:] Unicorn (page 294 of monster manual)
		\item[31-40:] Night elf(s)
		\item[41-45:] Ettin (page 132 of monster manual)
		\item[46-50:] troll (page 291 of monster manual)
		\item[51-55:] Galeb Duhr (page 139 of monster manual)
		\item[56-60:] Yeti (page 305 of monster manual)
		\item[61-65:] Ghost (page 147 of monster manual)
		\item[66-70:] Hook Horrer (page 189 of monster manual)
		\item[71-75:] Griffon (page 174 of monster manual)
		\item[76-80:] Hippogriff (page 184 of monster manual)
		\item[81-85:] Hell Hound (page 182 of monster manual)
		\item[86-90:] Jackalwere (page 193 of monster manual)
		\item[91-95:] Homunculus (page 188 of monster manual)
		\item[96-99:] Treant (page 289 of monster manual)
		\item[100:] Mythical Beast
	\end{description}
\end{commentbox}

\subsection{NPC's/Modified Creatures}

\begin{center}
	\includegraphics[width=\linewidth]{img/RexxarArt.jpg}
\end{center}
\begin{monsterbox}{Rexxar}
	\begin{hangingpar}
		\textit{Large humanoid, unaligned}
	\end{hangingpar}
	\dndline%
	\basics[%
	armorclass = 21,
	hitpoints  = 367,
	speed      = 50 ft
	]
	\dndline%
	\stats[
	STR = \stat{27}, % This stat command will autocomplete the modifier for you
	DEX = \stat{14},
	CON = \stat{25},
	INT = \stat{16},
	WIS = \stat{15},
	CHA = \stat{19}
	]
	\dndline%
	\details[%
	% If you want to use commas in these sections, enclose the
	% description in braces.
	% I'm so sorry.
	savingthrows = {Dex +9, Con +14, Wis +9, Char +11},
	skills = {Perception +16, Stealth +9},
	senses = {darkvision 90 ft., passive perception 26},
	languages = {common, Dwarvish},
	challenge = 21 (27500 XP)
	]
	\dndline%
	\begin{monsteraction}[Legendary Resistance (3/day)]
		If Rexxar fails a saving throw, he can choose to succeed instead.
	\end{monsteraction}	

	
	\monstersection{Actions}
	\begin{monsteraction}[Multiattack]
		Rexxar can make three attacks, one with each Axe and one with another weapon he has.
	\end{monsteraction}
	\begin{monsteraction}[Axe knock]
		Melee Weapon Attack: +15 to hit, reach 10ft., one target. Hit: 15 (2d6 + 8) bludgeoning damage. The target must succeed a DC 15 strength check or be knocked unconscious.
	\end{monsteraction}
	\begin{monsteraction}[Axe Slash]
		Melee Weapon Attack: +15 to hit, reach 10ft., one target. Hit: 18 (3d6 + 8) slashing damage.
	\end{monsteraction}
	\begin{monsteraction}[Axe Throw]
		Ranged Weapon Attack: +15 to hit, reach 35 ft., one target. Hit: 19 (2d10 + 8) piercing damage.
	\end{monsteraction}
	\begin{monsteraction}[Legendary Throw]
		Rexxar Drops one of his large Axes and hurls the other through the air with all his strength. This attack consumes all three of Rexxars attacks. Ranged Weapon Attack: +15 to hit, reach 45 ft., one target. Hit: 60 (10d10+10) damage. This attack will deal triple damage to a target that does not see it coming. Rexxar also gains 1 level of exhaustion from this attack.
	\end{monsteraction}
	\monstersection{Legendary Action}
	Rexxar can take 3 legendary actions per day.
	\begin{monsteraction}[Unbroken Will.]
		Rexxar Can use his sheer strength to free himself from any immobilizing effect or device.
	\end{monsteraction}
	\monstersection{Description/Information}
	Rexxar has lived in Rem Silva his entire life. He gets his strength from when he was a boy. Due to the effects of the energy stone on a fountain he drank out of, he is blessed with extraordinary strength. Along with his life of successful hunts, Rexxar has legendary strength and wit.
\end{monsterbox}

\begin{monsterbox}{Bella (Rexxar's Pet)}
	\begin{hangingpar}
		\textit{Large Beast (Bear), unaligned}
	\end{hangingpar}
	\dndline%
	\basics[%
	armorclass = 14,
	hitpoints  = 84,
	speed      = 40 ft
	]
	\dndline%
	\stats[
	STR = \stat{22}, % This stat command will autocomplete the modifier for you
	DEX = \stat{12},
	CON = \stat{18},
	INT = \stat{4},
	WIS = \stat{15},
	CHA = \stat{9}
	]
	\dndline%
	\details[%
	% If you want to use commas in these sections, enclose the
	% description in braces.
	% I'm so sorry.
	skills = Perception +5,
	senses = {passive perception 15},
	challenge = 3 (700XP)
	]
	\dndline%
	\begin{monsteraction}[Keen Smell]
		The bear has advantage on Wisdom (Perception) checks that rely on smell.
	\end{monsteraction}	
	
	\monstersection{Actions}
	\begin{monsteraction}[Multiattack]
		The bear makes two attacks: one with its bite and one with its claws.
	\end{monsteraction}
	\begin{monsteraction}[Bite]
		Melee Weapon Attack: +7 to hit, reach 5 ft., one target. Hit: 9 (ld8 + 5) piercing damage.
	\end{monsteraction}
	\begin{monsteraction}[Claws]
		Melee Weapon Attack: +7 to hit, reach 5 ft ., one target. Hit: 12 (2d6 + 5) slashing damage.
	\end{monsteraction}
	\monstersection{Description/Information}
		Bella was saved as a cub by Rexxar. Her parents were attacked by an ancient dinosaur that appeared due to the space stone. Rexxar raised Bella and trained her to follow commands and she has stuck by his side ever since.
\end{monsterbox}





\begin{monsterbox}{Displacer Beast}
	\begin{hangingpar}
		\textit{Large Monstrosity, lawful evil}
	\end{hangingpar}
	\dndline%
	\basics[%
	armorclass = 13,
	hitpoints  = 85,
	speed      = 40 ft
	]
	\dndline%
	\stats[
	STR = \stat{18}, % This stat command will autocomplete the modifier for you
	DEX = \stat{15},
	CON = \stat{16},
	INT = \stat{6},
	WIS = \stat{12},
	CHA = \stat{8}
	]
	\dndline%
	\details[%
	% If you want to use commas in these sections, enclose the
	% description in braces.
	% I'm so sorry.
	senses = {darkvision 60 ft., passive perception 11},
	challenge = 3 (700XP)
	]
	\dndline%
	\begin{monsteraction}[Avoidance]
		If the displacer beast is subjected to an effect that allows it to make a saving throw to take only half damage, it instead takes no damage if it succeeds on the saving throw and only half damage if it fails.
	\end{monsteraction}	
	\begin{monsteraction}[Displacement]
		The displacer projects a magical illusion that makes it appear to be standing near its actual location, causing attack tolls against it to have disadvantage. Due to the space stone effect on this beast, this trait is always active.
	\end{monsteraction}	
	\monstersection{Actions}
	\begin{monsteraction}[Multiattack]
		The displacer can make two attacks with its tentacles. 
	\end{monsteraction}
	\begin{monsteraction}[Tentacle]
		Melee Weapon Attack: +6 to hit, reach 10 ft., one target. Hit: 7 (1d6 +4) bludgeoning damage plus 3 (1s6) piercing damage.
	\end{monsteraction}	
	\monstersection{Description/Information}
	These creatures roam Rem Silva, but due to the effect of the space stones, they can fade in and out of reality (at random but not at will).
\end{monsterbox}

\begin{monsterbox}{Basilisk}
	\begin{hangingpar}
		\textit{Medium Monstrosity, unaligned}
	\end{hangingpar}
	\dndline%
	\basics[%
	armorclass = 15,
	hitpoints  = 52,
	speed      = 20 ft
	]
	\dndline%
	\stats[
	STR = \stat{16}, % This stat command will autocomplete the modifier for you
	DEX = \stat{8},
	CON = \stat{15},
	INT = \stat{2},
	WIS = \stat{8},
	CHA = \stat{7}
	]
	\dndline%
	\details[%
	% If you want to use commas in these sections, enclose the
	% description in braces.
	% I'm so sorry.
	senses = {darkvision 60 ft., passive perception 9},
	challenge = 3 (700XP)
	]
	\dndline%
	\begin{monsteraction}[Petrifying gaze]
		If a creature starts its turn within 30 ft and they can see each other, the basilisk can force a DC12 constitution saving throw (If the basilisk isn't incapacitated). One a fail, the greature begins to turn to stone and is restrained. It must repeat the saving throw at the end of its next turn. On a success, the effect ends. On a failure,  the creature is petrified until freed by the greater restoration spell or other magic.
		
		A creature not surprised can avert its eyes to avoid the saving throw at the start of its turn. If it does it cannot see the basilisk until the start of its next turn, when it can avert its eyes again. If it looks at the basilisk in the meantime, it must immediately make the save.
		
		If the basilisk sees it's reflection in bright light, it targets itself with it's gaze. 
	\end{monsteraction}	
	\begin{monsteraction}[Irregular Stone Skin]
		As a byproduct of the energy stones effect on the basilisk, enemies must roll at disadvantage if the basilisk succeeds a DC 12 strength save. Upon a failed attack agaisnt the basilisk, its skin hardens to absorb the impact of an attack.
	\end{monsteraction}	
	\monstersection{Actions}
	\begin{monsteraction}[Bite]
		Melee Weapon Attack: +5 to hit, reach 5 ft., one target. Hit 10 (2d6+3) piercing damage plus 7(2d6) poison damage.
	\end{monsteraction}
	\monstersection{Description/Information}
	These creatures roam around Rem Silva. Due to the effect of the energy stone, these creatures can be small or large with their stats adjusted accordingly. 
\end{monsterbox}

\begin{monsterbox}{Bullywug}
	\begin{hangingpar}
		\textit{Medium humanoid, neutral evil}
	\end{hangingpar}
	\dndline%
	\basics[%
	armorclass = {15 (hide armor, shield)},
	hitpoints  = 11 (2d8 +2),
	speed      = {20 ft, swim 40 ft.}
	]
	\dndline%
	\stats[
	STR = \stat{12}, % This stat command will autocomplete the modifier for you
	DEX = \stat{12},
	CON = \stat{13},
	INT = \stat{7},
	WIS = \stat{10},
	CHA = \stat{7}
	]
	\dndline%
	\details[%
	% If you want to use commas in these sections, enclose the
	% description in braces.
	% I'm so sorry.
	skills = {stealth +3}
	senses = {passive perception 10},
	challenge = 1/4 (50 XP)
	]
	\dndline%
	\begin{monsteraction}[Amphibious]
		The bullywug can breathe air and water.
	\end{monsteraction}	
	\begin{monsteraction}[Swamp Camouflage]
		The bullywug has advantage on Dexterity (stealth) checks made to hide in swampy terrain.
	\end{monsteraction}	
	\begin{monsteraction}[Standing Leap]
		The bullywug's long jump is up to 20 ft. and its high jump is up to 10 ft.
	\end{monsteraction}	

	\monstersection{Actions}
	\begin{monsteraction}[Multiattack]
		The bullywig makes two melee attacks: one with it's bite and one with its spear.
	\end{monsteraction}
	\begin{monsteraction}[Bite]
		Melee weapon attack: +3 to hit, reach 5 ft., one target. Hit: 3(1d4 +1) bludgeoning damage.
	\end{monsteraction}
	\begin{monsteraction}[Spear]
		Melee or ranged weapon attack: +3 to hit, reach 5 ft. or range 20/60., one target. Hit 4(1d6 +1) piercing damage or 5(1d8) piercing damage if used with two hands to make a melee attack. 
	\end{monsteraction}
	\monstersection{Description/Information}
		These creatures inhabit the river running through the center of Rem Silva. They have a camp at the southern end just before the opening to the sea.
\end{monsterbox}

\begin{monsterbox}{Ankylosaurus}
	\begin{hangingpar}
		\textit{Huges Beast, unaligned}
	\end{hangingpar}
	\dndline%
	\basics[%
	armorclass = 15,
	hitpoints  = 68,
	speed      = 30 ft
	]
	\dndline%
	\stats[
	STR = \stat{19}, % This stat command will autocomplete the modifier for you
	DEX = \stat{11},
	CON = \stat{15},
	INT = \stat{2},
	WIS = \stat{12},
	CHA = \stat{5}
	]
	\dndline%
	\details[%
	% If you want to use commas in these sections, enclose the
	% description in braces.
	% I'm so sorry.
	senses = {passive perception 11},
	challenge = 3 (700XP)
	]
	\dndline%	
	\monstersection{Actions}
	\begin{monsteraction}[Tail]
		Melee Weapon Attack: +7 to hit, reach 10 ft. one target. Hit: 18 (4d6 +4) bludgeoning damage. If the target is a creature, it must succeed a DC14 strength saving throw of be knocked prone.
	\end{monsteraction}
	\monstersection{Description/Information}
		Because od the time stone, prehistoric creatures like this can appear throughout Rem Silva.
\end{monsterbox}

\begin{monsterbox}{Night Elf Elite Warrior}
	\begin{hangingpar}
		\textit{Medium humanoid (elf), unaligned}
	\end{hangingpar}
	\dndline%
	\basics[%
	armorclass = 18,
	hitpoints  = 71,
	speed      = 30 ft
	]
	\dndline%
	\stats[
	STR = \stat{13}, % This stat command will autocomplete the modifier for you
	DEX = \stat{18},
	CON = \stat{14},
	INT = \stat{11},
	WIS = \stat{13},
	CHA = \stat{12}
	]
	\dndline%
	\details[%
	% If you want to use commas in these sections, enclose the
	% description in braces.
	% I'm so sorry.
	savingthrows = {Dex +7, Con +5, Wis +4},
	skills = {Perception +4, Stealth +10},
	senses = {darkvision 120 ft., passive perception 14},
	languages = {Elvish, undercommon, common},
	challenge = 5 (1800 XP)
	]
	\dndline%
	\begin{monsteraction}[Fey Ancestry]
		The elf has advantage on saving throws against being charmed, and magic can't put the elf to sleep.
	\end{monsteraction}	
	\begin{monsteraction}[Innate Spellcasting]
		The elfs spellcasting ability is Charisma (spell save DC 12). It can innately cast the following spells, requiring no material components:
		\begin{enumerate}
			\item At will: dancing lights
			\item 1/day each: darkness ,faerie fire , levitate (self only)
		\end{enumerate}
	\end{monsteraction}	
	\begin{monsteraction}[Sunlight Sensitivity]
		While in sunlight, the elf has disadvantage on attack rolls, as well as on Wisdom (Perception) checks that rely on sight.
	\end{monsteraction}	
	
	\monstersection{Actions}
	\begin{monsteraction}[Multiattack]
		The elf can make two shortsword attacks.
	\end{monsteraction}
	\begin{monsteraction}[Shortsword]
		Melee Weapon Attack: +7 to hit, reach 10ft., one target. Hit: 7 (ld6 + 4) piercing damage plus 10 (3d6) poison damage.
	\end{monsteraction}
	\begin{monsteraction}[Hand Crossbow]
		Ranged Weapon Attack: +7 to hit, range 30/120 ft ., one target. Hit: 7 (ld6 + 4) piercing damage, and the target must succeed on a DC 13 Constitution saving throw or	be poisoned for 1 hour. If the saving throw fails by 5 or more,	the target is also unconscious while poisoned in this way. The target wakes up if it takes damage or if another creature takes an action to shake it awake.
	\end{monsteraction}
	\monstersection{Reactions}
	\begin{monsteraction}[Parry]
		the elf adds 3 to its AC against one melee attack that would hit it. To do so, the elf must see the attacker and be
		wielding a melee weapon.
	\end{monsteraction}
	\monstersection{Description/Information}
		The night elves roam Rem Silva hiding in plain sight. They are the watchers of the forest. 
\end{monsterbox}

\begin{monsterbox}{Night Elf Elite Marksman}
	\begin{hangingpar}
		\textit{Medium humanoid (elf), unaligned}
	\end{hangingpar}
	\dndline%
	\basics[%
	armorclass = 18,
	hitpoints  = 71,
	speed      = 30 ft
	]
	\dndline%
	\stats[
	STR = \stat{12}, % This stat command will autocomplete the modifier for you
	DEX = \stat{19},
	CON = \stat{14},
	INT = \stat{11},
	WIS = \stat{13},
	CHA = \stat{13}
	]
	\dndline%
	\details[%
	% If you want to use commas in these sections, enclose the
	% description in braces.
	% I'm so sorry.
	savingthrows = {Dex +7, Con +5, Wis +4},
	skills = {Perception +4, Stealth +10},
	senses = {darkvision 120 ft., passive perception 14},
	languages = {Elvish, undercommon, common},
	challenge = 5 (1800 XP)
	]
	\dndline%
	\begin{monsteraction}[Fey Ancestry]
		The elf has advantage on saving throws against being charmed, and magic can't put the elf to sleep.
	\end{monsteraction}	
	\begin{monsteraction}[Innate Spellcasting]
		The elfs spellcasting ability is Charisma (spell save DC 12). It can innately cast the following spells, requiring no material components:
		\begin{enumerate}
			\item At will: dancing lights
			\item 1/day each: darkness ,faerie fire , levitate (self only)
		\end{enumerate}
	\end{monsteraction}	
	\begin{monsteraction}[Sunlight Sensitivity]
		While in sunlight, the elf has disadvantage on attack rolls, as well as on Wisdom (Perception) checks that rely on sight.
	\end{monsteraction}	
	
	\monstersection{Actions}
	\begin{monsteraction}[Multiattack]
		The elf can make two longbow attacks.
	\end{monsteraction}
	\begin{monsteraction}[Shortsword]
		Melee Weapon Attack: +7 to hit, reach 10ft., one target. Hit: 7 (ld6 + 4) piercing damage plus 10 (3d6) poison damage.
	\end{monsteraction}
	\begin{monsteraction}[Longbow]
		Ranged Weapon Attack: +7 to hit, range 30/120 ft ., one target. Hit: 10 (2d6 + 4) piercing damage, and the target must succeed on a DC 13 Constitution saving throw or	be poisoned for 1 hour. If the saving throw fails by 5 or more,	the target is also unconscious while poisoned in this way. The target wakes up if it takes damage or if another creature takes an action to shake it awake.
	\end{monsteraction}
	\monstersection{Reactions}
	\begin{monsteraction}[Skillful Avoidance]
		the elf adds 3 to its AC against one attack that would hit it. To do so, the elf must see the attacker and be
		wielding a longbow.
	\end{monsteraction}
	\monstersection{Description/Information}
	The night elves roam Rem Silva hiding in plain sight. They are the watchers of the forest. 
\end{monsterbox}

\begin{monsterbox}{Template}
	\begin{hangingpar}
		\textit{Medium Monstrosity, unaligned}
	\end{hangingpar}
	\dndline%
	\basics[%
	armorclass = 15,
	hitpoints  = 52,
	speed      = 20 ft
	]
	\dndline%
	\stats[
	STR = \stat{16}, % This stat command will autocomplete the modifier for you
	DEX = \stat{8},
	CON = \stat{15},
	INT = \stat{2},
	WIS = \stat{8},
	CHA = \stat{7}
	]
	\dndline%
	\details[%
	% If you want to use commas in these sections, enclose the
	% description in braces.
	% I'm so sorry.
	senses = {passive perception 9},
	challenge = 3 (700XP)
	]
	\dndline%
	\begin{monsteraction}[Petrifying gaze]
		
	\end{monsteraction}	
	
	\monstersection{Actions}
	\begin{monsteraction}[Bite]
		
	\end{monsteraction}
	\monstersection{Description/Information}
\end{monsterbox}

\section{The Pluvian Forest}

The Pluvian Forest is the name of the forest that is located to the North of Aurushire. The forest is known for strange spacial occurrences happening within it. Those who travel into The Pluvian Forest do not generally return or will return very confused or changed.

The Pluvian Forest is largely effected by the contents of the Spati Aethereu Thalamun (Aethereu). It is a normal forest in itself but it's close proximity to Aethereu makes this region dangerous. The forest is heavily affected by the space stone such that visitors can be lost for weeks while only traveling through a few days worth of terrain. Similarly, the time stone has the strongest connection to the space stone and thus has a great influence on the area. Often, travelers find the days lasting longer or shorter than usual. The energy stone has an effect on this region which amplifies the effect of the other two stones.

\begin{commentbox}{Irregular Days}
	As a byproduct of the time stone effecting the region, often the days find themselves to be shortened or lengthened due to the time stone effect from Aethereu. As a DM, you can determine periodically if there is any effect on travelers by rolling a 1d20 and succeeding a DC11 time throw. If failed, roll a 1d20 to determine the effect on the party.
	\hline
	\begin{description}
		\item[1:] Party transported 1 year into the future. 
		\item[2:] Party transported 3 months into the future. This may induce a season change. 
		\item[3:] Party transported 2 weeks into the future. This may induce a temperature change.
		\item[4-5:] Party transported 1 day into the future. 
		\item[6-7:] Party transported 5 hours into the future. 
		\item[8-10:] Party transported 1 hour into the future. 
		\item[11-13:] Party transported 1 hour into the past. 
		\item[14-15:] Party transported 5 hours into the past. 
		\item[16-17:] Party transported 1 day into the past. 
		\item[18:] Party transported 2 weeks into the past. This may induce a temperature change. 
		\item[19:] Party transported 3 months into the past. This may induce a season change. 
		\item[20:] Party transported 1 year into the past. 
	\end{description}
\end{commentbox}

\begin{commentbox}{Irregular Creatures}
	As a byproduct of the time stone working in conjunction with the space stone effecting the region, often creatures of objects of strange origin can appear in the area. As a DM, you can determine periodically if there is any effect on travelers by rolling a 1d20 and succeeding a DC13 space-time throw. If failed, roll a 1d20 to determine the effect on the party.
	\hline
	\begin{description}
		\item[1:] A prehistoric dinosaur appears in a nearby area.
		\item[2:] A long known-to-be extinct creature appears in a nearby area.
		\item[3:] A creature not native to this area appears in a nearby area.
		\item[4:] An ancient item appears in a nearby area.
		\item[5-6:] A creature native to the area appears behind the party.
		\item[7-10:] An item owned by a player vanishes and teleports to a location shortly behind them on their path.
		\item[11-14:] An item owned by a player vanishes and teleports to a location shortly ahead of them on their path.
		\item[15-16:] A creature native to the area appears ahead of the party.
		\item[17:] A futuristic item appears in a nearby area.
		\item[18:] A creature not native to this area appears in the nearby area.
		\item[19:] A natural creature that has never been seen before appears in the area.
		\item[20:] A robotic creature appears in the area.
	\end{description}
\end{commentbox}

\begin{commentbox}{Irregular Movement}
	As a byproduct of the space stone effecting the region, often the part finds themselves being moved around to different areas or places they have been before. As a DM, you can determine periodically if there is any effect on travelers by rolling a 1d20 and succeeding a DC15 space throw. If failed, roll a 1d20 to determine the effect on the party.
	\hline
	\begin{description}
		\item[1:] One member of the party is teleported to the entrance of The Pluvian Forest.
		\item[2:] The party members are teleported to a random location in The Pluvian Forest (chosen by the DM or completely random).
		\item[3:] The party is instantly moved to the last place they teleported from. If they have not been teleported yet, nothing will happen.
		\item[4:] Roll a DC12 save. Upon failing, the party is teleported to Rem Silvia.
		\item[5-6:] An object being carried by a member of the party is teleported just behind them on their path.
		\item[7-8:] The party is turned around.
		\item[9- 10:] The party is teleported to a place just ahead of where they weree. If they pass a DC15 perception check they will know they have moved. 
		\item[11-12:] The party is teleported to a place they recently were. If they pass a DC15 perception check they will know they have moved. The party may see tracks left by them which would lead them back to where they were.
		\item[13-14:] The party is turned around.
		\item[15-16:] An object being carried by a member of the party is teleported just ahead of them on their path. 
		\item[17:] Roll a DC12 save. Upon failing, the party is teleported to Aurushire.
		\item[18:] The party is instantly moved to the last place they teleported from. If they have not been teleported yet, nothing will happen.
		\item[19:] The party members are teleported to a random location in The Pluvian Forest (chosen by the DM or completely random).
		\item[20:] One member of the party is teleported to the end of The Pluvian Forest.
	\end{description}
\end{commentbox}

\begin{commentbox}{Irregular Energy}
	As a byproduct of the energy stone effecting the region, often creatures are either not as strong as they seem or have extraordinary strength. As a DM, you can determine periodically if there is any effect on travelers by rolling a 1d20 and succeeding a DC11 energy throw. If failed, roll a 1d20 to determine the effect on the party.
	\hline
	\begin{description}
		\item[1-5:] A member of the party acquires a level of exhaustion.
		\item[6:] A member of the party loses a spell slot.
		\item[7:] A member of the party loses 10 HP.
		\item[8-10:] A creature has all of its strength sapped and is very easy to defeat.
		\item[11-15:] Objects or areas of the forest glow and irradiate magical power. This can be trees, a stream, a pond, creatures, the ground, the path, or anything else. 
		\item[16-17:] A creature of the forest is bestowed with extraordinary strength (depending on party condition).
		\item[18:] A member of the party gains 10 HP. 
		\item[19:] A member of the party gains a missing spell slot. 
		\item[20:] A member of the party loses a level of exhaustion.
	\end{description}
\end{commentbox}

In order to successfully navigate through The Pluvian Forest and find Aethereu, the party must follow a simple set of instructions, while not getting turned around by the irregular occurrences. These set of instructions may be given to the party in a variety of ways (see below). 

\begin{commentbox}{Successful Navigation}
	\begin{description}
		\item[Guidence of Time] When nature calls, you must follow it's guidance. You must follow the hoot of the owls and the sounds of the wolves.
		\item[Guidence of Space] The correct path points to the stars. Follow the hills up and not down.
		\item[Guidence of Energy] The forest seeks to distract. Avoid illusions created by the energy stone.
		\item[Guidence of the Trinity] When the trinity is broken, search for the missing link. When two of the rules above are broken, look for the third to act.
	\end{description}
\end{commentbox}

\subsection{NPC's}
All of the creatures that can appear in The Pluvian forest are the same as those of Rem Silva Except the night elves and Rexxar/Bella.

\section{Spati Aethereu Thalamun}

Spati Aethereu Thalamun (Space chamber, also sometimes referred to as just Aethereu) is the name of one of the Three Trinity Stone chambers. Specifically, This one is the space chamber. It was created by The Space Stone with remnants of the time and matter stones. Not much is known about this chamber. It is presumed to have been made when the Incantation to hide the Trinity stones was finished. The only way to reach Spati Aethereu Thalamun is to successfully travel through The Pluvian Forest. The reasons for the strange occurrences within The Pluvian Forest are believed to originate from this chamber by those who know the stories of it.





\section{Auru Convallis}

This is the area to the East of Aurushire. This area is unexplored and all is known about it is that it is heavily dense with mountainous and forest terrain and the only known entrance path is through the southern shore which is blocked by a hostile Naga tribe.