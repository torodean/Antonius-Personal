\documentclass{article}

% Use the geometry package to adjust margins
\usepackage[margin=1in]{geometry}
\usepackage{multicol} % For two-column layout
\usepackage{circuitikz}
\usepackage{enumitem}
\usepackage{caption}
\usepackage{float} % Add the float package

% Remove section numbering and adjust formatting
\usepackage{titlesec}
\titleformat{\section}{\normalfont\Large\bfseries}{}{0em}{}[\titlerule]
\titleformat{\subsection}{\normalfont\large\bfseries}{}{0em}{}
\titleformat{\subsubsection}{\normalfont\normalsize\bfseries}{}{0em}{}

% Adjust the table of contents formatting
\usepackage{tocloft}
\renewcommand{\cftsecfont}{\normalfont}
\renewcommand{\cftsecpagefont}{\normalfont}
\renewcommand{\cftsecafterpnum}{\vspace{-0.5ex}}
\setlength{\cftbeforesecskip}{0.5ex}

\usepackage{lipsum} % For generating dummy text

\title{TESSA - Terrarium Environmental System with Smart Automation\\
	Documentation and Build Guide}
\author{Antonius Torode}
\date{\today}

\begin{document}
	
	\maketitle
	
	\tableofcontents
	
	%\listoffigures % If you have figures
	%\listoftables % If you have tables
	
	\newpage
	
	\begin{abstract}
		\lipsum[1]
	\end{abstract}
	
	\begin{multicols}{2} % Start two-column layout
		
		\section{Introduction}
		\lipsum[2-4]
		
		\section{Project Overview}
		\lipsum[5-7]
		
		\section{Components and Materials}
		
		At the heart of this project, some sort of micro-controller will be needed (i.e. Arduino Nano). This will store the basic timing and control components for the various components. These components include a temperature and humidity sensor (DHT11), a submersible low voltage water pump, and a display for output (OLED). the setup for this type of system is highly configurable and therefore the components can also vary widely. Basic circuit components will also be needed.
		
		For testing the system, jumper cables and a breadboard are recommended. A complete components list is available for two various/possible setups.
		
		\subsubsection{Arduino Setup Components List}
		\begin{itemize}[itemsep=1pt, parsep=1pt]
			\item 1 Arduino Nano
			\item 1 DHT11 Sensor
			\item 1 SSD1306 OLED Display
			\item 1 micro 5V submersible pump
			\item 1 10k$\Omega$ resistor
			\item 1 diode (1N4001-1N4007)
			\item 1 2N 2222A Transistor (or similar NPN transistor)
			\item 1 12$\times$24 PCB
			\item Wires (varying sizes)
			\item Soldering kit (Solder Iron, Solder, Flux, etc)
		\end{itemize}
		
		\subsubsection{Raspberry Pi Setup Components List}
		\begin{itemize}[itemsep=1pt, parsep=1pt]
			\item 1 Raspberry pi
			\item 1 DHT11 Sensor
			\item 1 SSD1306 OLED Display (optional)
			\item 1 micro 5V submersible pump
			\item 1 10k$\Omega$ resistor
			\item Wires (varying sizes)
			\item Jumper cables
		\end{itemize}
		
		
		\subsection{Component Descriptions}
			
			\subsubsection{DHT11}
			
			The DHT11 sensor is a low-cost and easy to use sensor for measuring temperature and humidity levels. The sensor provides digital output, making it easy to interface with micro-controllers like Arduino or Raspberry Pi. It uses a single wire to transmit two environmental data points in a single package. The DHT11 has the following features:			
			\begin{itemize}[itemsep=1pt, parsep=1pt]
				\item 3 to 5V power and I/O
				\item 2.5mA max current use during conversion (while requesting data)
				\item Good for 20-80\% humidity readings with 5\% accuracy
				\item Good for 0-50$^\circ$C temperature readings $\pm$2$^\circ$C accuracy
				\item No more than 1 Hz sampling rate (once every second)
				\item Body size 15.5mm $\times$ 12mm $\times$ 5.5mm
				\item 4 pins with 0.1" spacing
			\end{itemize}			
			
			\begin{minipage}{0.85\columnwidth} % Create a minipage that spans a single column
				\begin{figure}[H] % Specify placement options (here, 'htbp' means 'here', 'top', 'bottom', 'page')
					\centering % Center the figure horizontally
					\begin{circuitikz}
						% Draw the rectangle for DHT11
						\draw (0,0) rectangle (2,2) node[midway] {DHT11};
						
						% Draw pins on the right side
						\draw (2,1.7) -- ++(1,0) -- ++(0,1) node[draw, circle, minimum size=0.5cm, anchor=south] {5V}; % VCC
						\draw (2,1) -- ++(4,0) node[draw, circle, minimum size=0.5cm, anchor=west] {S}; % Data		
						\draw (5,1) -- (5,1.7) to [R, l=$10 \, \mathrm{k}\Omega$] (3,1.7);
						\draw (2,0.3) -- ++(1,0) node[ground] {}; % GND
					\end{circuitikz}
					\caption{\footnotesize The DHT11 circuit setup. The GND pin needs connected to ground. The 5V pin needs connected to a constant 5V DC source. The data pin needs connected to the proper data channel to read from. A $10\mathrm{k}\Omega$ resistor is connected between the data pin and voltage pin that serves as a `pull-up' for the data.}
					\label{fig:DHT11}
				\end{figure}
			\end{minipage}
	
			\subsubsection{Mini Submersible Water Pump}
			
			The pump I found for this project was a micro submersible mini water pump. It has the following features:			
			\begin{itemize}[itemsep=1pt, parsep=1pt]
				\item Rated voltage: 3.3V or 5V DC
				\item No load of water discharge capacity: 100L / H
				\item Load rated current: 150-250mA, Use: diving type
			\end{itemize}
			
			\begin{minipage}{0.85\columnwidth} % Create a minipage that spans a single column
				\begin{figure}[H] % Specify placement options (here, 'htbp' means 'here', 'top', 'bottom', 'page')
					\centering % Center the figure horizontally
					\begin{circuitikz}
						% Draw the rectangle for DHT11
						\draw (0,0) rectangle (2,1) node[midway] {Pump};
						
						% Draw pins on the right side
						\draw (2,0.7) -- ++(1.5,0) node[draw, circle, minimum size=0.5cm, anchor=west] {5V}; % VCC
						\draw (2,0.3) -- ++(1,0) node[ground] {}; % GND
					\end{circuitikz}
					\caption{\footnotesize The water pump only has two connectors, one for voltage, and the other for ground.}
					\label{fig:pump}
				\end{figure}
			\end{minipage}
		
			\subsubsection{SSD1306 I$^2$C OLED Display}
			
			The display I used for this project is a SSD1306 0.96 inch I$^2$C organic light-emitting diode (OLED) display. The OLED display doesn’t require backlight and the pixels only consume energy when on. The model I am using has 4 pins ($V_{in}$, GND, SCL, SDA). Some features of the display are as follows:
			\begin{itemize}[itemsep=1pt, parsep=1pt]
				\item Resolution: 128 x 64 dot matrix panel
				\item Support voltage: 3.3V-5V DC
				\item Wide range of operating temperature: -40$^\circ$C to 85$^\circ$C
				\item Embedded Driver IC: SSD1306. Communication: I2C/IIC Interface, only need two I / O ports
			\end{itemize}
		
			The I$^2$C driver means the bus consists of two signals: SDA and SCL. SDA (Serial Data) is the data signal and SCL (Serial Clock) is the clock signal.
		
			\begin{minipage}{0.85\columnwidth} % Create a minipage that spans a single column
				\begin{figure}[H] % Specify placement options (here, 'htbp' means 'here', 'top', 'bottom', 'page')
					\centering % Center the figure horizontally
					\begin{circuitikz}
						% Draw the rectangle for DHT11
						\draw (0,0) rectangle (2,2) node[midway] {OLED};
						
						% Draw pins on the right side
						\draw (0.3, 2) -- ++(0,0.5) -- ++(-0.5, 0) node[draw, circle, minimum size=0.5cm, anchor=east] {5V}; % VCC
						\draw (0.7, 2) -- ++(0,1) node[ground, rotate=180] {}; % GND
						\draw (1.3, 2) -- ++(0,1) -- ++(0.5,0) node[draw, circle, minimum size=0.5cm, anchor=west] {SCL}; % VCC
						\draw (1.7, 2) -- ++(0,0.25) -- ++(1,0) node[draw, circle, minimum size=0.5cm, anchor=west] {SDA}; % GND
					\end{circuitikz}
					\caption{\footnotesize The OLED display has 4 inputs. The VCC is the power input, GND is ground, SDA is the serial data connector and SCL is the serial clock.}
					\label{fig:OLED}
				\end{figure}
			\end{minipage}
		
			\subsubsection{Arduino Nano}
			
			The Arduino Nano is a compact and versatile micro-controller board designed for embedded electronics projects and prototyping. The Nano is known for its small physical size, making it suitable for applications with limited space or when a compact design is essential. The Nano includes a set of digital and analog pins, allowing you to interface with sensors, actuators, and other electronic components. These pins can be programmed for input or output. It features a USB connector for easy programming and communication with a computer. This makes it convenient for uploading code and debugging. It is programmed using the Arduino IDE, which provides a user-friendly interface for writing, uploading, and debugging code. A large community and extensive libraries are available to simplify development. Like other Arduino boards, the Nano is open-source hardware and software. This means that the design files, schematics, and software are freely available for modification and customization.
			
			Some features of the arduino nano that are important for this project are as follows:
			\begin{itemize}[itemsep=1pt, parsep=1pt]
				\item High-performance low-power 8-bit processor
				\item Has a 5V DC output pin.
				\item Has multiple programmable I/O pins.
				\item Has SDA and SCL connections for I$^2$C communication.
			\end{itemize}
		
			\subsubsection{Raspberry Pi}
			
			The Raspberry Pi is a versatile and powerful single-board computer designed for various computing and electronics projects. The Raspberry Pi is equipped with a high-performance ARM-based\footnote{This is important because the newer versions of the arduino IDE are not pre-built for the ARM architecture.} processor (varies by model) that provides the processing power needed for a wide range of applications. It features a set of General-Purpose Input/Output (GPIO) pins, which can be programmed for digital input and output as well as hardware interfacing. These pins are crucial for connecting sensors, actuators, and other components. The Raspberry Pi supports communication protocols like I$^2$C and SPI, allowing you to connect and communicate with a variety of sensors and devices easily. Like Arduino, the Raspberry Pi is open-source hardware and software. It has a large and active community that offers support, tutorials, and a wide range of software packages.
			
			Some Raspberry Pi models include built-in Wi-Fi and Bluetooth capabilities, making wireless communication and IoT projects convenient. The Raspberry Pi has an HDMI output for connecting to displays, making it suitable for applications that require visual feedback. Between these two features, a lot of possibilities open up with allowing the DHT11 sensor to be displayed on Pi screen rather than an OLED as well as potentially being able to monitor it remotely.
			
			Some features of the raspberry pi has that are important for this project are as follows:
			\begin{itemize}[itemsep=1pt, parsep=1pt]
				\item Has a 5V DC output pin.
				\item Has multiple programmable GPIO pins.
				\item Has SDA and SCL connections for I$^2$C communication.
			\end{itemize}
		
		\section{System Architecture}
		THIS SECTION IN PROGRESS.
		
		\section{Raspberry Pi Implementation}
		
			\subsection{Circuit Diagram and Wiring}
			THIS SECTION IN PROGRESS.
			
			\subsection{Software Setup}
			THIS SECTION IN PROGRESS.
			
			\subsection{Python Code}
			THIS SECTION IN PROGRESS.
			
			\subsection{Testing and Troubleshooting}
			THIS SECTION IN PROGRESS.
		
		\section{Arduino Nano Implementation}
		THIS SECTION IN PROGRESS.
		
			\subsection{Circuit Diagram and Wiring}
			THIS SECTION IN PROGRESS.
			
			\subsection{Arduino Code}
			THIS SECTION IN PROGRESS.
			
			\subsection{Testing and Troubleshooting}
			THIS SECTION IN PROGRESS.
		
		\section{3D Printed Enclosure}
		THIS SECTION IN PROGRESS.
		
		\section{Power Supply}
		THIS SECTION IN PROGRESS.
		
		\section{Usage and Operation}
		THIS SECTION IN PROGRESS.
		
		\section{Maintenance and Troubleshooting}
		THIS SECTION IN PROGRESS.
		
		\section{Future Enhancements}
		THIS SECTION IN PROGRESS.
		
		\section{Conclusion}
		THIS SECTION IN PROGRESS.
	
		\section{Author Contact Information}
		
\end{multicols} % End two-column layout	

\begin{thebibliography}{X}
	\bibitem{DHT11} DHT11: https://www.adafruit.com/product/386
	\bibitem{SSD1306_datasheet} SSD1306 Datasheet: https://cdn-shop.adafruit.com/datasheets/SSD1306.pdf
	\bibitem{arduino nano} Arduino Nano: https://docs.arduino.cc/hardware/nano
\end{thebibliography}	
	
		\section{Appendices}
		THIS SECTION IN PROGRESS.
		
		\subsection{Appendix A: Bill of Materials}
		
		\subsection{Appendix B: Schematics and PCB Designs}
		
		\subsection{Appendix C: Code Samples}
		
		\subsection{Appendix D: 3D Printing Files}
	
\end{document}
