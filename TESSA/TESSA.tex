\documentclass{article}

% Use the geometry package to adjust margins
\usepackage[margin=1in]{geometry}
\usepackage{multicol} % For two-column layout
\usepackage{circuitikz}
\usepackage{enumitem}
\usepackage{caption}
\usepackage{float} % Add the float package

% Remove section numbering and adjust formatting
\usepackage{titlesec}
\titleformat{\section}{\normalfont\Large\bfseries}{}{0em}{}[\titlerule]
\titleformat{\subsection}{\normalfont\large\bfseries}{}{0em}{}
\titleformat{\subsubsection}{\normalfont\normalsize\bfseries}{}{0em}{}

% Adjust the table of contents formatting
\usepackage{tocloft}
\renewcommand{\cftsecfont}{\normalfont}
\renewcommand{\cftsecpagefont}{\normalfont}
\renewcommand{\cftsecafterpnum}{\vspace{-0.5ex}}
\setlength{\cftbeforesecskip}{0.5ex}

\usepackage{lipsum} % For generating dummy text

\title{TESSA - Terrarium Environmental System with Smart Automation\\
	Documentation and Build Guide}
\author{Antonius Torode}
\date{\today}

\begin{document}
	
	\maketitle
	
	\tableofcontents
	
	\listoffigures % If you have figures
	\listoftables % If you have tables
	
	\begin{abstract}
		\lipsum[1]
	\end{abstract}
	
	\begin{multicols}{2} % Start two-column layout
		
		\section{Introduction}
		\lipsum[2-4]
		
		\section{Project Overview}
		\lipsum[5-7]
		
		\section{Components and Materials}
		\lipsum[8-10]
		
		\subsection{Component Descriptions}
			
			\subsubsection{DHT11}
			
			The DHT11 sensor is a low-cost and easy to use sensor for measuring temperature and humidity levels. The sensor provides digital output, making it easy to interface with micro-controllers like Arduino or Raspberry Pi. It uses a single wire to transmit two environmental data points in a single package. The DHT11 has the following features:			
			\begin{itemize}[itemsep=1pt, parsep=1pt]
				\item 3 to 5V power and I/O
				\item 2.5mA max current use during conversion (while requesting data)
				\item Good for 20-80\% humidity readings with 5\% accuracy
				\item Good for 0-50$^\circ$C temperature readings $\pm$2$^\circ$C accuracy
				\item No more than 1 Hz sampling rate (once every second)
				\item Body size 15.5mm $\times$ 12mm $\times$ 5.5mm
				\item 4 pins with 0.1" spacing
			\end{itemize}			
			
			\begin{minipage}{0.85\columnwidth} % Create a minipage that spans a single column
				\begin{figure}[H] % Specify placement options (here, 'htbp' means 'here', 'top', 'bottom', 'page')
					\centering % Center the figure horizontally
					\begin{circuitikz}
						% Draw the rectangle for DHT11
						\draw (0,0) rectangle (2,2) node[midway] {DHT11};
						
						% Draw pins on the right side
						\draw (2,1.7) -- ++(1,0) -- ++(0,1) node[draw, circle, minimum size=0.5cm, anchor=south] {5V}; % VCC
						\draw (2,1) -- ++(4,0) node[draw, circle, minimum size=0.5cm, anchor=west] {S}; % Data		
						\draw (5,1) -- (5,1.7) to [R, l=$10 \, \mathrm{k}\Omega$] (3,1.7);
						\draw (2,0.3) -- ++(1,0) node[ground] {}; % GND
					\end{circuitikz}
					\caption{\footnotesize The DHT11 circuit setup. The GND pin needs connected to ground. The 5V pin needs connected to a constant 5V DC source. The data pin needs connected to the proper data channel to read from. A $10\mathrm{k}\Omega$ resistor is connected between the data pin and voltage pin that serves as a `pull-up' for the data.}
					\label{fig:DHT11}
				\end{figure}
			\end{minipage}
	
			\subsubsection{Mini Submersible Water Pump}
			
			The pump I found for this project was a micro submersible mini water pump. It has the following features:			
			\begin{itemize}[itemsep=1pt, parsep=1pt]
				\item Rated voltage: 3.3V or 5V DC
				\item No load of water discharge capacity: 100L / H
				\item Load rated current: 150-250mA, Use: diving type
			\end{itemize}
			
			\begin{minipage}{0.85\columnwidth} % Create a minipage that spans a single column
				\begin{figure}[H] % Specify placement options (here, 'htbp' means 'here', 'top', 'bottom', 'page')
					\centering % Center the figure horizontally
					\begin{circuitikz}
						% Draw the rectangle for DHT11
						\draw (0,0) rectangle (2,1) node[midway] {Pump};
						
						% Draw pins on the right side
						\draw (2,0.7) -- ++(1.5,0) node[draw, circle, minimum size=0.5cm, anchor=west] {5V}; % VCC
						\draw (2,0.3) -- ++(1,0) node[ground] {}; % GND
					\end{circuitikz}
					\caption{\footnotesize The water pump only has two connectors, one for voltage, and the other for ground.}
					\label{fig:pump}
				\end{figure}
			\end{minipage}
		
			\subsubsection{SSD1306 I$^2$C OLED Display}
			
			The display I used for this project is a SSD1306 0.96 inch I$^2$C organic light-emitting diode (OLED) display. The OLED display doesn’t require backlight and the pixels only consume energy when on. The model I am using has 4 pins ($V_{in}$, GND, SCL, SDA). Some features of the display are as follows:
			\begin{itemize}[itemsep=1pt, parsep=1pt]
				\item Resolution: 128 x 64 dot matrix panel
				\item Support voltage: 3.3V-5V DC
				\item Wide range of operating temperature: -40$^\circ$C to 85$^\circ$C
				\item Embedded Driver IC: SSD1306. Communication: I2C/IIC Interface, only need two I / O ports
			\end{itemize}
		
			The I$^2$C driver means the bus consists of two signals: SDA and SCL. SDA (Serial Data) is the data signal and SCL (Serial Clock) is the clock signal.
		
			\begin{minipage}{0.85\columnwidth} % Create a minipage that spans a single column
				\begin{figure}[H] % Specify placement options (here, 'htbp' means 'here', 'top', 'bottom', 'page')
					\centering % Center the figure horizontally
					\begin{circuitikz}
						% Draw the rectangle for DHT11
						\draw (0,0) rectangle (2,2) node[midway] {OLED};
						
						% Draw pins on the right side
						\draw (0.3, 2) -- ++(0,0.5) -- ++(-0.5, 0) node[draw, circle, minimum size=0.5cm, anchor=east] {5V}; % VCC
						\draw (0.7, 2) -- ++(0,1) node[ground, rotate=180] {}; % GND
						\draw (1.3, 2) -- ++(0,1) -- ++(0.5,0) node[draw, circle, minimum size=0.5cm, anchor=west] {SCL}; % VCC
						\draw (1.7, 2) -- ++(0,0.25) -- ++(1,0) node[draw, circle, minimum size=0.5cm, anchor=west] {SDA}; % GND
					\end{circuitikz}
					\caption{\footnotesize The OLED display has 4 inputs. The VCC is the power input, GND is ground, SDA is the serial data connector and SCL is the serial clock.}
					\label{fig:OLED}
				\end{figure}
			\end{minipage}
		
		\section{System Architecture}
		\lipsum[14-16]
		
		\section{Raspberry Pi Implementation}
		
			\subsection{Circuit Diagram and Wiring}
			\lipsum[20-22]
			
			\subsection{Software Setup}
			\lipsum[23-25]
			
			\subsection{Python Code}
			\lipsum[26-28]
			
			\subsection{Testing and Troubleshooting}
			\lipsum[29-31]
		
		\section{Arduino Nano Implementation}
		\lipsum[32-34]
		
		\subsection{Circuit Diagram and Wiring}
		\lipsum[35-37]
		
		\subsection{Arduino Code}
		\lipsum[38-40]
		
		\subsection{Testing and Troubleshooting}
		\lipsum[41-43]
		
		\section{3D Printed Enclosure}
		\lipsum[44-46]
		
		\section{Power Supply}
		\lipsum[47-49]
		
		\section{Usage and Operation}
		\lipsum[50-52]
		
		\section{Maintenance and Troubleshooting}
		\lipsum[53-55]
		
		\section{Future Enhancements}
		\lipsum[56-58]
		
		\section{Conclusion}
		\lipsum[59-61]	
		
		\begin{thebibliography}{X}
			\bibitem{DHT11} DHT11: https://www.adafruit.com/product/386
			\bibitem{SSD1306_datasheet} SSD1306 Datasheet: https://cdn-shop.adafruit.com/datasheets/SSD1306.pdf
		\end{thebibliography}	
	
		\section{Appendices}
		\lipsum[65-67]
		
		\subsection{Appendix A: Bill of Materials}
		
		\subsection{Appendix B: Schematics and PCB Designs}
		
		\subsection{Appendix C: Code Samples}
		
		\subsection{Appendix D: 3D Printing Files}
		
		\section{Author Contact Information}
		
	\end{multicols} % End two-column layout
	
\end{document}
