

\documentclass[12pt]{article}

%%% These are some packages that are useful
\usepackage{amsfonts, lipsum}
\usepackage{amsmath,amssymb, amscd,amsbsy, amsthm, enumerate}
\usepackage{mdframed, titlesec, setspace,verbatim, multicol}
\usepackage[top=1in, bottom=1in, left=.45in, right=.45in]{geometry}
\usepackage[unicode]{hyperref}
\usepackage{tikz, pgfplots, xcolor, fancyhdr}
\usepackage{listings}
\usepackage{xcolor}
\usepackage{textcomp}

%%% Page formatting
%\setlength{\headsep}{30pt}
\setlength{\parindent}{0pt}
\setlength{\textheight}{9in}

%%% Header and Footer Info
\pagestyle{fancy}
\fancyhead[L]{{\large \textbf{Practice - 001}}}
\fancyhead[C]{Date:}
\fancyhead[R]{Name: \space\space\space\space\space\space\space\space\space\space\space\space\space\space\space\space\space\space\space\space\space\space\space}
\fancyfoot[L]{\LaTeX}
\fancyfoot[C]{}
\fancyfoot[R]{\thepage}

%%% These define our question environment and help number things correctly
\theoremstyle{definition}
\newtheorem{thm}{Theorem}
\newtheorem{question}[thm]{Question}
\newtheorem{prop}[thm]{Proposition}
\newtheorem{lem}[thm]{Lemma}
\newtheorem{DEF}[thm]{Definition}
\newtheorem{rem}[thm]{Remark}

%%% This defines the solution environment for you to write your solutions
\newenvironment{soln}
{\let\oldqedsymbol=\qedsymbol
	\renewcommand{\qedsymbol}{$ $}
	\begin{proof}[\bfseries\upshape \color{blue}Solution]\color{blue}}
	{\end{proof}
	\renewcommand{\qedsymbol}{\oldqedsymbol}}

%%% This defines the note environment for you to write your notes
\newenvironment{note}
{\let\oldqedsymbol=\qedsymbol
	\renewcommand{\qedsymbol}{$ $}
	\begin{proof}[\bfseries\upshape \color{blue}Note]\color{Red}}
	{\end{proof}
	\renewcommand{\qedsymbol}{\oldqedsymbol}}


%%% These are some shortcuts that are handy
\def\real{{\mathbb R}}
\def\Natural{\mathbb{N}}
\def\dx{\textnormal{dx}}
\def\dy{\textnormal{dy}}
\def\dz{\textnormal{dz}}
\def\dt{\textnormal{dt}}
\def\ds{\textnormal{ds}}
\def\dw{\textnormal{dw}}
\def\Re{\textnormal{Re}}
\def\Im{\textnormal{Im}}
\def\exp{\textnormal{exp}}
\def\interior{\textnormal{interior}}
\def\al{\alpha}
\def\del{\delta}
\def\Del{\Delta}
\def\gam{\gamma}
\def\Gam{\Gamma}
\def\Om{\Omega}
\def\ep{\varepsilon}
\def\lam{\lambda}
\def\rational{{\mathbb Q}}
\def\integer{{\mathbb Z}}
\def\Q{{\mathbb Q}}
\def\Z{{\mathbb Z}}
\def\N{{\mathbb N}}
\def\R{{\mathbb R}}
\def\grad{\nabla}
\def\C{\mathcal C}
\def\P{\mathcal P}
\def\T{\mathcal T}
\def\I{\mathcal I}
\newcommand{\abs}[1]{\left| #1 \right|}
\newcommand{\inner}[1]{\langle #1 \rangle}
\newcommand{\norm}[1]{\left\lVert#1\right\rVert}
\newcommand{\spanvect}{\textnormal{span}}
\newcommand{\union}{\cup}
\newcommand{\Union}{\bigcup}
\def\intersect{\cap}
\def\Intersect{\bigcap}

\DeclareMathOperator*{\Limsup}{LIMSUP}




%%% Document Starts now
\begin{document}

	\begin{center}
	{\LARGE \textbf{2$^{\textrm{nd}}$ grade Problem Solving: 001}}\\
		\vspace{.6cm}
	\end{center}
	
\begin{question}
	List ten items that are found in most (almost all) houses.
\end{question}

\vspace{4cm}

\begin{question}
	Sally is exactly three feet tall. She knows that each foot has twelve inches and knows that john is two inches taller than her. How tall is John in inches?
\end{question}

\vspace{4cm}

\begin{question}
	Find the next two terms of the following sequence:
	\begin{large}
	\begin{align*}
	1,1,2,3,5,8,13,21, 34,\dots
	\end{align*}
	\end{large}
\end{question}

\vspace{3cm}

\begin{question}
	Find the three missing numbers in the sequence below:
	\begin{large}
	\begin{align*}
	\frac{1}{2}, \frac{2}{3}, \frac{3}{ }, \frac{4}{5}, \frac{5}{ }, \frac{ }{7}, \frac{7}{8}
	\end{align*}
	\end{large}
\end{question}

\newpage

\begin{question}
David is heading to the mall. He first stops at a shoe shop and buys two shoes. Then, he heads to a game store and buys one board game and two card games. Afterwards, he heads to a media store and buys three movies, one CD, and headphones. next, he heads to the food court and buys a hamburger and a water for lunch. Finally, he heads to the book store and buys one book right before leaving. How many items did David buy?
\end{question}

\vspace{2cm}

\begin{question}
	Solve the following and show all your work:
	\begin{align*}
	1+2+3+4+5+6+7+8+9+10+11+12+13+14+15+16+17+18+19
	\end{align*}
\end{question}

\vspace{4cm}

\begin{question}
	Find and explain the mistake in the following story.
	\begin{quote}
		Tippy was a very small dog. She loved to go for long walks. Some days she would cook herself lunch then go for a walk. She loved to walk down the road and all over the neighborhood. Some days she would even go for walks when the sun was already down. 
	\end{quote}
\end{question}

\newpage

\begin{question}
	Unscramble the following combinations of letters to make words.
	\begin{multicols}{2}\begin{large}
	\begin{align*}
	&\textrm{elohl} \\
	&\textrm{ogd} \\
	&\textrm{pzzai} \\
	&\textrm{etbal} 
	\end{align*}	
	
	\begin{align*}
	&\textrm{lferow} \\
	&\textrm{utmocepr} \\
	&\textrm{ath} \\
	&\textrm{iplowl} 
	\end{align*}
	
	\end{large}\end{multicols}
\end{question}

\begin{question}
	Sally's father has 5 children. Their names are John, Jack, Jeff, and Josh. What is the name of the fifth child?
\end{question}

\vspace{2cm}

\begin{question}
	Place the following objects in order from heaviest to lightest.
	\begin{large}
	\begin{center}
		a bowling ball, a basketball, the moon, the sun, \\ the earth, a marble, a helium balloon
	\end{center}
	\end{large}
\end{question}



%%%%%%%%%%%%%%%%%%%%%%%%%%%%%%%%%%%%%%%%%%%%%%%%%%%%%%%%%%%%%%%%%%%%%%%%%%%%%%%%%%%%%%%%%%%

%%%%%%%%%%%%%%%%%%%%%%%%%%%%%%%%%%%%%%%%%%%%%%%%%%%%%%%%%%%%%%%%%%%%%%%%%%%%%%%%%%%%%%%%%%%
\end{document}





















