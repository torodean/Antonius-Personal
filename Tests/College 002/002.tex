

\documentclass[12pt]{article}

%%% These are some packages that are useful
\usepackage{amsfonts, lipsum}
\usepackage{amsmath,amssymb, amscd,amsbsy, amsthm, enumerate}
\usepackage{mdframed, titlesec, setspace,verbatim, multicol}
\usepackage[top=1in, bottom=1in, left=.45in, right=.45in]{geometry}
\usepackage[unicode]{hyperref}
\usepackage{tikz, pgfplots, xcolor, fancyhdr}
\usepackage{listings}
\usepackage{xcolor}

%%% Page formatting
%\setlength{\headsep}{30pt}
\setlength{\parindent}{0pt}
\setlength{\textheight}{9in}

%%% Header and Footer Info
\pagestyle{fancy}
\fancyhead[L]{{\large Practice - \textbf{002}}}
\fancyhead[C]{Date:}
\fancyhead[R]{Name: \space\space\space\space\space\space\space\space\space\space\space\space\space\space\space\space\space\space\space\space\space\space\space}
\fancyfoot[L]{\LaTeX}
\fancyfoot[C]{}
\fancyfoot[R]{\thepage}

%%% These define our question environment and help number things correctly
\theoremstyle{definition}
\newtheorem{thm}{Theorem}
\newtheorem{question}[thm]{Question}
\newtheorem{prop}[thm]{Proposition}
\newtheorem{lem}[thm]{Lemma}
\newtheorem{DEF}[thm]{Definition}
\newtheorem{rem}[thm]{Remark}

%%% This defines the solution environment for you to write your solutions
\newenvironment{soln}
{\let\oldqedsymbol=\qedsymbol
	\renewcommand{\qedsymbol}{$ $}
	\begin{proof}[\bfseries\upshape \color{blue}Solution]\color{blue}}
	{\end{proof}
	\renewcommand{\qedsymbol}{\oldqedsymbol}}

%%% This defines the note environment for you to write your notes
\newenvironment{note}
{\let\oldqedsymbol=\qedsymbol
	\renewcommand{\qedsymbol}{$ $}
	\begin{proof}[\bfseries\upshape \color{blue}Note]\color{Red}}
	{\end{proof}
	\renewcommand{\qedsymbol}{\oldqedsymbol}}


%%% These are some shortcuts that are handy
\def\real{{\mathbb R}}
\def\Natural{\mathbb{N}}
\def\dx{\textnormal{dx}}
\def\dy{\textnormal{dy}}
\def\dz{\textnormal{dz}}
\def\dt{\textnormal{dt}}
\def\ds{\textnormal{ds}}
\def\dw{\textnormal{dw}}
\def\Re{\textnormal{Re}}
\def\Im{\textnormal{Im}}
\def\exp{\textnormal{exp}}
\def\interior{\textnormal{interior}}
\def\al{\alpha}
\def\del{\delta}
\def\Del{\Delta}
\def\gam{\gamma}
\def\Gam{\Gamma}
\def\Om{\Omega}
\def\ep{\varepsilon}
\def\lam{\lambda}
\def\rational{{\mathbb Q}}
\def\integer{{\mathbb Z}}
\def\Q{{\mathbb Q}}
\def\Z{{\mathbb Z}}
\def\N{{\mathbb N}}
\def\R{{\mathbb R}}
\def\grad{\nabla}
\def\C{\mathcal C}
\def\P{\mathcal P}
\def\T{\mathcal T}
\def\I{\mathcal I}
\newcommand{\abs}[1]{\left| #1 \right|}
\newcommand{\inner}[1]{\langle #1 \rangle}
\newcommand{\norm}[1]{\left\lVert#1\right\rVert}
\newcommand{\spanvect}{\textnormal{span}}
\newcommand{\union}{\cup}
\newcommand{\Union}{\bigcup}
\def\intersect{\cap}
\def\Intersect{\bigcap}

\DeclareMathOperator*{\Limsup}{LIMSUP}



\lstset { %
	language=C++,
	backgroundcolor=\color{black!5}, % set backgroundcolor
	basicstyle=\footnotesize,% basic font setting
}

%%% Document Starts now
\begin{document}

	\begin{center}
	{\LARGE \textbf{Entry College Level Problem Solving: 002}}\\
		\vspace{.6cm}
	\end{center}
	\begin{center}
		Please write all answers on this page. You may use scrap paper to do your work on but please attach any scrap paper used after completion. Do not use a calculator.
	\end{center}
	
	The limit definition of the derivative of a function $f: \real \rightarrow \real$ is given by the formula
	\begin{align}
	\frac{\partial}{\partial x}f(x) = f'(x)= \lim\limits_{\Delta x \rightarrow 0} \frac{f(x+\Delta x)-f(x)}{\Delta x}
	\end{align}
\begin{question}
	Compute the derivative of the function $g: \real \rightarrow \real$ where $g(x)=x^2+2x+1$ using the limit definition.
\end{question}

\vspace{3 cm}

\begin{question}
	Compute the derivative of the function $h: \real \rightarrow \real$ where $h(x)=x^3+5x$ using the limit definition.
\end{question}

\vspace{3cm}


\begin{question}
	On planet X, the height of an object at some time $t$ that is thrown upwards at some initial velocity $v_0$ is given by $h(t)=v_0t-30t^2$. In order to determine the velocity $v(t)$ and acceleration $a(t)$ we can use
	\begin{align}
	v(t)&=h'(t) \\
	a(t)&=h''(t) = v'(t).
	\end{align}
	Calculate the velocity and acceleration functions on planet X.
\end{question}


\newpage
The quadratic formula is used to determine the roots (zeros) of any polynomial expression of the form $0=ax^2+bx+c$ and is given by
\begin{align}
x=\frac{-b\pm\sqrt{b^2-4ac}}{2a}
\end{align}
\begin{question}
	Find the roots of $4=2x^2+3x+5$.
\end{question}

\vspace{3cm}

\begin{question}
	Find an equation for the roots of $c-2b=cx^2+4ab-2cx$ in simplest form.
\end{question}

\vspace{3cm}

In order to quickly calculate the sum (addition) of all values from $1$ to $n$, we can use the formula
\begin{align}
\sum_{i=1}^{n}i=\frac{n(n+1)}{2}
\end{align}
\begin{question}
	Using the above formula, calculate the sum of all natural numbers from 1 to 500.
\end{question}

\vspace{3cm}

\begin{question}
	Using equation (5) from above, determine the sum of all natural numbers from 1 to 60 plus the sum of all natural numbers from 80 to 120.
\end{question}

\newpage
Similar to the previous summation equation, the sum of the square of all natural numbers from 1 to $n$ can be determined by
\begin{align}
\sum_{i=1}^{n}i^2=\frac{n(n+1)(2n+1)}{6}.
\end{align}
And furthermore
\begin{align}
\sum_{i=1}^{n}i^3=\frac{n^2(n+1)^2}{4}.
\end{align}
\begin{question}
	Use equations (5)-(7) to calculate 
	\begin{align}
	\sum_{i=1}^{15}i^3+i^2+i
	\end{align}
\end{question}

\vspace{2cm}

\begin{question}
	Use equations (5)-(7) to calculate 
	\begin{align}
	\sum_{i=1}^{10}4i^3+2i^2+2i
	\end{align}
\end{question}

\vspace{2cm}
	\begin{question}
Solve the following system of equations for $x,y,z$.
\begin{align}
2x+7y+3z&=14 \\
4x+5z+2y&=5 \\
z+9x+y&=2
\end{align}
\end{question}

%%%%%%%%%%%%%%%%%%%%%%%%%%%%%%%%%%%%%%%%%%%%%%%%%%%%%%%%%%%%%%%%%%%%%%%%%%%%%%%%%%%%%%%%%%%

%%%%%%%%%%%%%%%%%%%%%%%%%%%%%%%%%%%%%%%%%%%%%%%%%%%%%%%%%%%%%%%%%%%%%%%%%%%%%%%%%%%%%%%%%%%
\end{document}





















